\section{Conclus�es}
\label{sec:concl}

Neste trabalho apresentou-se m�todos para identifica��o de sistemas utilizando-se
m�todos param�tricos, em primeiro lugar estudou-se a estimativa da resist�ncia de
um resistor, por diversas formas e como que o ruido interfere em cada um destes m�todos
e qual � a vantagem de um m�todo sobre o outro.

Em uma segunda etapa estudou-se a aplica��o do m�todo de regress�o linear utilizando como
crit�rio de otimiza��o o m�todo dos m�nimos quadrados. Utilizando-se este m�todo foi
identificado um sistema com ruido, aplicando-se diferentes entradas ao mesmo e observando 
como que isso influencia sobre a identifica��o do sistema em conjunto com o ruido 
presente no mesmo.

Apresentou-se um comparativo do m�todo para diferentes amplitudes de ruido sobre o sistema
e tamb�m comparou-se os resultados para entradas n�o aleat�rias. chegou-se a conclus�o que
entradas aleat�rias conseguem excitar mais o sistema e produzir melhores resultados que 
ondas com menos compelentes de frequ�ncia (senoides por exemplo).


