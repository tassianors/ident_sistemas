\section{Conclus�es}
\label{sec:concl}

Na se��o (\ref{sec:arx}) obteve-se uma estimativa para o sistema (\ref{eq:system}), mesmo 
que o modelo ARX utilizado (\ref{eq:model_arx}) n�o conseguisse representar o sistema 
completamente. Desta forma obteve-se estimativas polarizadas para os par�metros $a$ e $b$, 
como observado nas Figuras (\ref{fig:arx_sin_pi4}) e (\ref{fig:arx_sin_pi20}), al�m de que
a informa��o estimada para o ruido n�o era representativa com a realidade, j� que para o modelo
ARX considera-se que o ruido � submetido a influencia dos mesmos polos da fun��o de transfer�ncia
$G(q)$.

Para resolver este grave problema da polariza��o foi utilizado o m�todo dos m�nimos quadrados
generalizado, onde utilizando-se outro estimador (\ref{eq:estimador}) para sistema, obteve-se
resultados bem mais promissores, que podem ser observados nas  Figuras (\ref{fig:gmq_ab}) e 
(\ref{fig:gmq_cd}), com este algoritmo, foi poss�vel considerar um modelo completo para representar 
o sistema, o que trouxe a estimativa alem dos par�metros de $G(q)$ tamb�m da fun��o $H(q)$.

Neste trabalho observamos que quando a fam�lia de modelos n�o representa o sistema real, tem-se
erro de polariza��o das estimativas efetuadas. Desta forma, mesmo utilizando mais pontos para
a simula��o e fazendo-se mais simula��es, os par�metros estimados na m�dia n�o chegam ao valor
real do sistema. Foi apontado um algoritmo para resolu��o de problemas onde o modelo padr�o dos 
m�nimos quadrados (ARX) n�o pode ser utilizado, e os resultados foram apresentados.
