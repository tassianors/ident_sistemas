\section{M�todo dos m�nimos quadrados}
\label{sec:mmq}
%===============================================================================

O m�todo dos m�nimos quadrados (MMQ) � um dos mais conhecidos e mais utilizados 
nas mais diversas �reas da ci�ncia e tecnologia. A origem da ideia b�sica pode ser
encontrada nos trabalhos de Gaus sobre o estudo astron�micos. \cite{aguirre}

\subsection{Sistema com solu��o �nica}

Considerando-se que o sistema que ser� observado seja linear e invariante no tempo.
Se a fun��o $f$ que descreve o sistema for n�o linear o sistema poder� em principio 
ser identificado por modelos n�o lineares. Com base nestas restri��es temos que:

\begin{equation}
\begin{matrix}
\begin{matrix}
\begin{bmatrix}
y_1\\ 
y_2\\ 
\vdots \\ 
y_n
\end{bmatrix} = &
\begin{bmatrix}
x_1 & x_2 & \cdots  & x_n
\end{bmatrix} &
\begin{bmatrix}
\theta_1\\ 
\theta_2\\ 
\vdots \\ 
\theta_n
\end{bmatrix}
\end{matrix}
\\ \\
y=X\theta
\end{matrix}
\nonumber
\end{equation}

Com $X \in \Re^{nxn}$. Desde que $X$ seja n�o singular � poss�vel determinar $\theta$:

\begin{equation}
\theta=X^{-1}y
\label{eq:mmq_base}
\end{equation}

Para sistemas sobredeterminados onde $ N > n$, A vari�vel $X$ da equa��o (\ref{eq:mmq_base}) fica
$X \in \Re^{Nxn}$. Como esta matriz n�o � quadrada, n�o � poss�vel de ser invertida. Multiplicando-se
a equa��o (\ref{eq:mmq_base}) por $X^T$ tem-se: \cite{aguirre}

\begin{equation}
X^Ty=X^TX\theta
\nonumber
\end{equation}

De onde vem:

\begin{equation}
\theta = [X^T X]^{-1} X^T y
\label{eq:mmq}
\end{equation}

O m�todo dos m�nimos quadrados minimiza o crit�rio apresentado em (\ref{eq:mmq_j}).

\begin{equation}
J(\theta)=\frac{1}{2N}\sum_{t=1}^{N}[y(t)-\hat{y}(t, \theta)]^2
\label{eq:mmq_j}
\end{equation}

Onde $\hat{y}(t, \theta)$ � a predi��o do sistema e pode ser representado como abaixo:

\begin{equation}
\hat{y}(t, \theta)=\theta^T \phi(t)
\nonumber
\end{equation}

Desta forma pode se dizer que o sistema real � o pr�prio sistema estimado mais algum 
erro de estimativa:

\begin{equation}
y(t) = \hat{y}(t, \theta) +e(t)=\theta^T \phi(t) + e(t)
\nonumber
\end{equation}


