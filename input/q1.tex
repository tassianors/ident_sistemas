\section{Quest�o 1}
\label{sec:q1}
%===============================================================================

Quest�o: verificar de o LSE (estimador de m�nimos quadrados) produz estimativas 
n�o polarizadas no caso de estrituras do tipo ARX ARMAX e OE.

Genericamente, modelos utilizados para identifica��o de sistemas podem ser representados
pela equa��o (\ref{eq:q1_model}).

\begin{equation}
A(z, \theta)Y(t)=\frac{B(z, \theta)}{F(z, \theta)}U(t)+\frac{C(z, \theta)}{D(z, \theta)}e(t)
\label{eq:q1_model}
\end{equation}

Onde:

\begin{equation}
\begin{matrix}
A(z, \theta)=1+a_1 Z^{-1}+a_2 Z^{-2}+\cdots +a_{na} Z^{-na}\\
B(z, \theta)=b_1 Z^{-1}+b_2 Z^{-2}+\cdots +b_{nb} Z^{-nb}\\ 
C(z, \theta)=1+c_1 Z^{-1}+c_2 Z^{-2}+\cdots +c_{nc} Z^{-nc}\\ 
D(z, \theta)=1+d_1 Z^{-1}+d_2 Z^{-2}+\cdots +d_{na} Z^{-nd}\\ 
F(z, \theta)=1+f_1 Z^{-1}+f_2 Z^{-2}+\cdots +f_{nf} Z^{-nf} 
\end{matrix}
\nonumber
\end{equation}

Baseado nestas informa��es existem modelos onde apenas alguns destes polin�mios s�o 
diferentes de 1. Na Tabela (\ref{tab:model}) s�o apresentados alguns destes modelos
mais comumente utilizados.

\begin{table}[htbp]
  \begin{center}
	\caption{Modelos comumente utilizados para identifica��o de sistemas}
	\label{tab:model}
	\begin{small}
	  \begin{tabular}{rc}
		\hline
		Modelo & Polin�mios diferentes de 1 \\
		\hline
		FIR	& B \\
		ARX	& A B \\
		ARMAX & A B C \\
		ARMA & A C \\
		ARARMAX & A B C D \\
		OE & B F \\
		BJ & B F C D \\
		\hline
	  \end{tabular}
	\end{small}
  \end{center}
\end{table}


\subsection{ARX}
\label{sec:q1_arx}
%===============================================================================

A partir de (\ref{eq:q1_model}) tem-se que o modelo ARX fica (\ref{eq:q1_model_arx}).

\begin{equation}
A(z, \theta)Y(t)=B(z, \theta)U(t)+e(t)
\label{eq:q1_model_arx}
\end{equation}

A equa��o (\ref{eq:q1_model_arx}) pode ser reescrita :

\begin{equation}
\begin{matrix}
Y(t)=b_1 Z^{-1}U(t)+b_2 Z^{-2}U(t)+\cdots +b_{nb} Z^{-nb}U(t)\\
	 -a_1 Z^{-1}Y(t)-b_2 Z^{-2}Y(t)-\cdots -a_{na} Z^{-na}Y(t) + e(t)
\end{matrix}
\nonumber
\end{equation}

O que pode ser escrito como em (\ref{eq:q1_yt}).

\begin{equation}
Y(t)=\varphi ' \theta +e(t)
\label{eq:q1_yt}
\end{equation}

Onde:

\begin{equation}
\begin{matrix}
\theta = \begin{bmatrix}
a_1\\ 
\vdots \\ 
a_{na}\\ 
b_1\\ 
\vdots \\ 
b_{nb}
\end{bmatrix}
 & 
\varphi (t)=\begin{bmatrix}
y(t-1)\\ 
\vdots \\ 
y(t-na)\\ 
u(t-1)\\ 
\vdots \\ 
u(t-nb)
\end{bmatrix}
\\ 
& \\
\Phi = \begin{bmatrix}
\varphi '(1)\\ 
\varphi '(2)\\ 
\vdots\\ 
\varphi '(N)
\end{bmatrix} & 
\end{matrix}
\nonumber
\end{equation}

Pode -se ent�o encontrar a estimativa de $\theta$, ou seja $\hat{\theta}$.

\begin{equation}
\begin{matrix}
\hat{\theta}=(\Phi' \Phi)^{-1}(\Phi'(\Phi \theta_0+e))=\theta_0+(\Phi' \Phi)^{-1}\Phi'e\\ 
E\{ \hat{\theta} \}=\theta_0+(\Phi' \Phi)^{-1}\Phi'E\{ e(t)\}
\end{matrix}
\nonumber
\end{equation}

Mas como $E\{ e(t)\}=0$, pois o ruido possui m�dia zero. A estimativa de $\hat{\theta}$ � 
o pr�prio $\theta$.

De forma semelhante pode-se escrever:

\begin{equation}
\begin{matrix}
E\{(\hat{\theta}-\theta)(\hat{\theta}-\theta)' \}=E\{((\Phi'\Phi)^{-1}\Phi'e)((\Phi'\Phi)^{-1}\Phi'e)' \}
\\
=(\Phi'\Phi)^{-1}\Phi' E\{ e\; e'\}\Phi(\Phi'\Phi)^{-1}
\\ 
=E\{(\hat{\theta}-\theta)(\hat{\theta}-\theta)' \}=conv(\hat{\theta})=\lambda^2(\Phi'\Phi)^{-1}\\
pois : E\{ e\; e'\} = \lambda^2 I
\end{matrix}
\nonumber
\end{equation}

J� que a convolu��o da estimativa $\hat{\theta}$ corresponde ao valor encontrado, o
erro de polariza��o para o sistema ARX � igual a zero.


\subsection{ARMAX}
\label{sec:q1_armax}
%===============================================================================

A partir de (\ref{eq:q1_model}) tem-se que o modelo ARMAX fica (\ref{eq:q1_model_armax}).

\begin{equation}
A(z)Y(t)=B(z, \theta)U(t)+C(z, \theta)e(t)
\label{eq:q1_model_armax}
\end{equation}
