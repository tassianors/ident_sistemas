\section{M�todo da vari�vel instrumental - IV}
\label{sec:iv}
%===============================================================================

Uma alternativa para a minimizac�o da polarizac�o PPA (propriedades de pequenas
amostras) na estimativa do sistema � a polarizac�o assintotica. A id�ia � relaxar
um pouco a definic�o PPA e, por um lado, permitir que haja polarizac�o para uma
amostra pequena, mas por outro lado, verificar se tal polarizac�o desaparece � 
medida que o tamanho do conjunto de observac�es cresce. \cite{aguirre}

Para utilizac�o deste m�todo, esclhe-se um Instrumento $Z(t)$:

\begin{equation}
Z(t) \in \Re^{p}\; \forall t \;\;\; E(Z(t)\nu)=0
\label{eq:iv_instrumento}
\end{equation}

\begin{equation}
\hat{y}(t, \theta)=\phi ^T (t)\theta
\nonumber
\end{equation}

\begin{equation}
E[Z(t)(y(t)-\hat{y}(t, \theta)]=0
\nonumber
\end{equation}

\begin{equation}
E[Z(t)\phi^T (t)]\theta = E[Z(t)y(t)]
\nonumber
\end{equation}

De onde vem:

\begin{equation}
\hat{\theta}_{N}^{iv}=[\sum_{t=1}^{N}Z(t)\phi^T(t)]^{-1}[\sum_{t=1}^{N}Z(t)y(t)]
\label{eq:iv_estim}
\end{equation}

Ap�s a escolha da estimativa $Z(t)$ que satisfaca (\ref{eq:iv_estim}) o passo seguinte 
� calcular (\ref{eq:iv_estim_2}).

\begin{equation}
E[\hat{\theta}_{N}^{iv}-\theta_0]=0
\label{eq:iv_estim_2}
\end{equation}


\subsection{M�todo aplicado ao controle de posic�o}
%===============================================================================

Primeiro passo para aplicar o m�todo das {\it{Vari�veis instrumentais}} � escolher
o instrumento que ser� utilizado.

\begin{equation}
\begin{matrix}
w(t)=F(q)u(t)\\
F(q)=q^{-1}\\ 
w(t)=u(t-1)
\end{matrix}
\nonumber
\end{equation}

O que resulta em um instrumento apresentado em (\ref{eq:iv_motor_inst}).

\begin{equation}
Z(t)=\begin{bmatrix}
w(t)\\ 
w(t-1)\\ 
\vdots \\
w(t-p)
\end{bmatrix}
\label{eq:iv_motor_instr}
\end{equation}


