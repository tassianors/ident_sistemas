%===============================================================================
% ABSTRACT
%===============================================================================

Neste trabalho ser� apresentado diversos meios para a identifica��o de sistemas
lineares. Existem dois grupos principais de m�todos para esta identifica��o, sendo
um deles conhecido como {\it{identifica��o n�o param�trica}} onde existem infinitos
par�metros para serem estimados e que normalmente � utilizado para identifica��o de
fun��es gr�ficas. Outro m�todo � conhecido como {\it{identifica��o param�trica}} onde
o n�mero de par�metros a ser estimado � finito. Este �ltimo m�todo ser� abordado
neste trabalho, por possuir uma aplicabilidade maior e a possibilidade de estimar 
processos em fun��es matem�ticas que descrevem o comportamento do sistema muitas vezes
com mais informa��o que os m�todos gr�ficos.

Para este trabalho ser� utilizado um processo de controle de posi��o angular, controlado
por um motor de corrente continua. Ser�o apresentados dois m�todos para identifica��o
do sistema: m�nimos quadrados e vari�veis instrumentais. A fim de compara��o, ser�
identificado o sistema utilizando-se um modelo que n�o consegue descrever o sistema
de forma completa.

Ao fim ser� apresentado uma breve analise qualitativa dos resultados obtidos no 
decorrer das estimativas.

