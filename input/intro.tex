%===============================================================================
\section{Introdu��o}

Neste trabalho ser� apresentado um sistema de para controle de posi��o angular,
manipulado por um motor de corrente continua (DC). O objetivo principal, � estimar
os valores das vari�veis existentes no modelo escolhido para representar este sistema.

Inicialmente ser� explicado o processo de escolha do modelo que representa a din�mica
deste sistema (Se��o (\ref{sec:modelling})). Ser� explicitado quais considera��es sobre
o sistema foram feitas para se obter o modelo que ser� utilizado nas se��es seguintes, 
para determinar os par�metros.

Em seguida, ser� utilizado o m�todo dos M�nimos quadrados (MQ), para estimar o sistema, 
considerando-se para isso que o ruido sobre o sistema sofre influ�ncia dos mesmos polos
que est�o na planta, ou seja, que o modelo para o sistema se comporta como um modelo ARX.
Nesta mesma se��o (\ref{sec:mmq}) ser� apresentado os resultados para o mesmo sistema, 
baseado nos mesmos dados, mas para um modelo que n�o representa o sistema f�sico, ou que 
n�o consegue representa-lo.

Na se��o (\ref{sec:iv}) ser� apresentado o m�todo das vari�veis instrumentais, para estimar
os valores do par�metro para o modelo. Da mesma forma que para o m�todo dos MQ, ser�
utilizado um modelo que n�o representa o sistema real, e este m�todo ser� aplicado para 
determinar a qualidade dos resultados obtidos.

Ao fim, ser� apresentado uma breve discuss�o sobre os resultados obtidos em ambos os m�todos
utilizados, e as considera��es finais.

